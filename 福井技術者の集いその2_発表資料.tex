\documentclass{jarticle}%
\usepackage{amsmath}
\usepackage{amsfonts}
\usepackage{amssymb}
\usepackage{graphicx}%
\setcounter{MaxMatrixCols}{30}
%TCIDATA{OutputFilter=latex2.dll}
%TCIDATA{Version=5.50.0.2952}
%TCIDATA{CSTFile=jarticle55.cst}
%TCIDATA{Created=Tuesday, March 25, 2014 19:08:12}
%TCIDATA{LastRevised=Saturday, February 07, 2015 11:25:17}
%TCIDATA{<META NAME="GraphicsSave" CONTENT="32">}
%TCIDATA{<META NAME="SaveForMode" CONTENT="1">}
%TCIDATA{BibliographyScheme=Manual}
%TCIDATA{<META NAME="DocumentShell" CONTENT="Standard LaTeX\jshell\Blank - Japanese Article[jarticle]">}
%BeginMSIPreambleData
\providecommand{\U}[1]{\protect\rule{.1in}{.1in}}
%EndMSIPreambleData
\newtheorem{theorem}{\U{5b9a}\U{7406}}
\newtheorem{acknowledgement}[theorem]{\U{627f}\U{8a8d}}
\newtheorem{algorithm}[theorem]{\U{30a2}\U{30eb}\U{30b4}\U{30ea}\U{30ba}\U{30e0}}
\newtheorem{axiom}[theorem]{\U{516c}\U{7406}}
\newtheorem{case}[theorem]{\U{5834}\U{5408}}
\newtheorem{claim}[theorem]{\U{4e3b}\U{5f35}}
\newtheorem{conclusion}[theorem]{\U{7d50}\U{8ad6}}
\newtheorem{condition}[theorem]{\U{6761}\U{4ef6}}
\newtheorem{conjecture}[theorem]{\U{63a8}\U{8ad6}}
\newtheorem{corollary}[theorem]{\U{7cfb}}
\newtheorem{criterion}[theorem]{\U{57fa}\U{6e96}}
\newtheorem{definition}[theorem]{\U{5b9a}\U{7fa9}}
\newtheorem{example}[theorem]{\U{4f8b}}
\newtheorem{exercise}[theorem]{\U{7df4}\U{7fd2}}
\newtheorem{lemma}[theorem]{\U{88dc}\U{984c}}
\newtheorem{notation}[theorem]{\U{8a18}\U{53f7}}
\newtheorem{problem}[theorem]{\U{554f}\U{984c}}
\newtheorem{proposition}[theorem]{\U{547d}\U{984c}}
\newtheorem{remark}[theorem]{\U{8a18}\U{4e8b}}
\newtheorem{solution}[theorem]{\U{89e3}\U{6cd5}}
\newtheorem{summary}[theorem]{\U{8981}\U{7d04}}
\newenvironment{proof}[1][\U{8a3c}\U{660e}]{\noindent\textbf{#1.} }{\ \rule{0.5em}{0.5em}}
\begin{document}
%
\begin{tabular}
[c]{ll}%
$-\dfrac{1}{2}\Delta\psi_{i}\left(  \vec{r}\right)  +V\left(  r\right)
\psi_{i}\left(  \vec{r}\right)  =E\psi_{i}\left(  \vec{r}\right)  ,$ &
$V\left(  r\right)  =-\dfrac{1}{r}$%
\end{tabular}


$\Delta=\dfrac{\partial^{2}}{\partial r^{2}}+\dfrac{2}{r}\dfrac{\partial
}{\partial r}+\dfrac{1}{r^{2}}\left[  \dfrac{1}{\sin\theta}\dfrac{\partial
}{\partial\theta}\left(  \sin\theta\dfrac{\partial}{\partial\theta}\right)
+\dfrac{1}{\sin^{2}\theta}\dfrac{\partial^{2}}{\partial\phi^{2}}\right]  $

$\psi_{i}\left(  \vec{r}\right)  =r^{l}L_{nl}\left(  r\right)  Y_{lm}\left(
\theta,\phi\right)  $

$\left\{
\begin{tabular}
[c]{l}%
$\dfrac{d^{2}L_{nl}\left(  r\right)  }{dr^{2}}+\dfrac{2\left(  l+1\right)
}{r}\dfrac{dL_{nl}\left(  r\right)  }{dr}=2\left[  V\left(  r\right)
-E\right]  L_{nl}\left(  r\right)  $\\
$\hat{l}^{2}Y_{lm}\left(  \theta,\phi\right)  =l\left(  l+1\right)
Y_{lm}\left(  \theta,\phi\right)  $%
\end{tabular}
\ \right.  $

$\dfrac{d^{2}L_{nl}\left(  r\right)  }{dr^{2}}+\dfrac{2\left(  l+1\right)
}{r}\dfrac{dL_{nl}\left(  r\right)  }{dr}=2\left[  V\left(  r\right)
-E\right]  L_{nl}\left(  r\right)  $


\end{document}